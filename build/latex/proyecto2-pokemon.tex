%% Generated by Sphinx.
\def\sphinxdocclass{report}
\documentclass[letterpaper,10pt,spanish,openany,oneside]{sphinxmanual}
\ifdefined\pdfpxdimen
   \let\sphinxpxdimen\pdfpxdimen\else\newdimen\sphinxpxdimen
\fi \sphinxpxdimen=.75bp\relax

\PassOptionsToPackage{warn}{textcomp}
\usepackage[utf8]{inputenc}
\ifdefined\DeclareUnicodeCharacter
% support both utf8 and utf8x syntaxes
  \ifdefined\DeclareUnicodeCharacterAsOptional
    \def\sphinxDUC#1{\DeclareUnicodeCharacter{"#1}}
  \else
    \let\sphinxDUC\DeclareUnicodeCharacter
  \fi
  \sphinxDUC{00A0}{\nobreakspace}
  \sphinxDUC{2500}{\sphinxunichar{2500}}
  \sphinxDUC{2502}{\sphinxunichar{2502}}
  \sphinxDUC{2514}{\sphinxunichar{2514}}
  \sphinxDUC{251C}{\sphinxunichar{251C}}
  \sphinxDUC{2572}{\textbackslash}
\fi
\usepackage{cmap}
\usepackage[T1]{fontenc}
\usepackage{amsmath,amssymb,amstext}
\usepackage{babel}



\usepackage{times}
\expandafter\ifx\csname T@LGR\endcsname\relax
\else
% LGR was declared as font encoding
  \substitutefont{LGR}{\rmdefault}{cmr}
  \substitutefont{LGR}{\sfdefault}{cmss}
  \substitutefont{LGR}{\ttdefault}{cmtt}
\fi
\expandafter\ifx\csname T@X2\endcsname\relax
  \expandafter\ifx\csname T@T2A\endcsname\relax
  \else
  % T2A was declared as font encoding
    \substitutefont{T2A}{\rmdefault}{cmr}
    \substitutefont{T2A}{\sfdefault}{cmss}
    \substitutefont{T2A}{\ttdefault}{cmtt}
  \fi
\else
% X2 was declared as font encoding
  \substitutefont{X2}{\rmdefault}{cmr}
  \substitutefont{X2}{\sfdefault}{cmss}
  \substitutefont{X2}{\ttdefault}{cmtt}
\fi


\usepackage[Sonny]{fncychap}
\ChNameVar{\Large\normalfont\sffamily}
\ChTitleVar{\Large\normalfont\sffamily}
\usepackage{sphinx}

\fvset{fontsize=\small}
\usepackage{geometry}

% Include hyperref last.
\usepackage{hyperref}
% Fix anchor placement for figures with captions.
\usepackage{hypcap}% it must be loaded after hyperref.
% Set up styles of URL: it should be placed after hyperref.
\urlstyle{same}
\addto\captionsspanish{\renewcommand{\contentsname}{Contents:}}

\usepackage{sphinxmessages}
\setcounter{tocdepth}{1}



\title{Proyecto2-Pokemon}
\date{09 de diciembre de 2019}
\release{}
\author{Doggos}
\newcommand{\sphinxlogo}{\vbox{}}
\renewcommand{\releasename}{}
\makeindex
\begin{document}

\ifdefined\shorthandoff
  \ifnum\catcode`\=\string=\active\shorthandoff{=}\fi
  \ifnum\catcode`\"=\active\shorthandoff{"}\fi
\fi

\pagestyle{empty}
\sphinxmaketitle
\pagestyle{plain}
\sphinxtableofcontents
\pagestyle{normal}
\phantomsection\label{\detokenize{index::doc}}


\begin{sphinxadmonition}{note}{Nota:}
Bienvenido a la documentación del Proyecto 2 de la materia de Redes de Computadoras
\end{sphinxadmonition}


\chapter{¿Cómo usar?}
\label{\detokenize{index:como-usar}}
\begin{sphinxadmonition}{note}{Nota:}
Primero inicializamos el servidor, y después los clientes pueden iniciar una conexión
\end{sphinxadmonition}
\begin{itemize}
\item {} 
Para el servidor
\begin{itemize}
\item {} 
Si es la primera vez usando el servidor del juego, por favor dirigirse a la carpeta \sphinxstyleemphasis{./Instalaciones} y ejecutar \sphinxstyleemphasis{./make\_servidor}

\item {} 
En caso contrario: en una terminal, nos situamos en la ubicación del archivo \sphinxstyleemphasis{pokemonServer.py}

\item {} 
No necesitamos parámetros extra para ejecutar el servidor. \sphinxstyleemphasis{./pokemonServer.py}

\end{itemize}

\item {} 
Para el cliente
\begin{itemize}
\item {} 
Si es la primera vez usando un cliente del juego, por favor dirigirse a la carpeta \sphinxstyleemphasis{./Instalaciones} y ejecutar \sphinxstyleemphasis{./make\_cliente}

\item {} 
En una terminal, nos situamos en la ubicación del archivo \sphinxstyleemphasis{pokemonClient.py}

\item {} 
Este programa recibe como parámetros iniciales la dirección IP a través de la cual se quiere conectar, y el puerto. Por lo tanto, ejecutamos de la siguiente manera: \sphinxstyleemphasis{./pokemonClient.py \textless{}IP\textgreater{} \textless{}port\textgreater{}}

\end{itemize}

\end{itemize}

\begin{sphinxadmonition}{note}{Nota:}
El puerto ingresado al ejecutar el cliente debe ser el mismo que usa el servidor
\end{sphinxadmonition}


\chapter{Programas involucrados}
\label{\detokenize{index:programas-involucrados}}

\section{Cliente Pokemon Go!}
\label{\detokenize{pokemonClient:cliente-pokemon-go}}\label{\detokenize{pokemonClient::doc}}
Implementación de un cliente para el juego Pokemon Go! e interactúa directamente con el usuario

\phantomsection\label{\detokenize{pokemonClient:module-pokemonClient}}\index{pokemonClient (módulo)@\spxentry{pokemonClient}\spxextra{módulo}}\index{cerrarSesion() (en el módulo pokemonClient)@\spxentry{cerrarSesion()}\spxextra{en el módulo pokemonClient}}

\begin{fulllineitems}
\phantomsection\label{\detokenize{pokemonClient:pokemonClient.cerrarSesion}}\pysiglinewithargsret{\sphinxcode{\sphinxupquote{pokemonClient.}}\sphinxbfcode{\sphinxupquote{cerrarSesion}}}{\emph{soc}}{}
Cierre normal de sesión del usuario.
\begin{quote}\begin{description}
\item[{Parámetros}] \leavevmode
\sphinxstyleliteralstrong{\sphinxupquote{soc}} (\sphinxstyleliteralemphasis{\sphinxupquote{Socket}}) \textendash{} Socket de la conexión

\item[{Devuelve}] \leavevmode
Nada

\end{description}\end{quote}

\end{fulllineitems}

\index{displayCatalogo() (en el módulo pokemonClient)@\spxentry{displayCatalogo()}\spxextra{en el módulo pokemonClient}}

\begin{fulllineitems}
\phantomsection\label{\detokenize{pokemonClient:pokemonClient.displayCatalogo}}\pysiglinewithargsret{\sphinxcode{\sphinxupquote{pokemonClient.}}\sphinxbfcode{\sphinxupquote{displayCatalogo}}}{\emph{catalogo}}{}~\begin{description}
\item[{Imprime en pantalla el catálogo de Pokemones}] \leavevmode
disponibles de manera «amigable».

\end{description}
\begin{quote}\begin{description}
\item[{Parámetros}] \leavevmode
\sphinxstyleliteralstrong{\sphinxupquote{catalogo}} (\sphinxstyleliteralemphasis{\sphinxupquote{List of String}}) \textendash{} Catalogo de Pokemones

\item[{Devuelve}] \leavevmode
Nada

\end{description}\end{quote}

\end{fulllineitems}

\index{displayPokedex() (en el módulo pokemonClient)@\spxentry{displayPokedex()}\spxextra{en el módulo pokemonClient}}

\begin{fulllineitems}
\phantomsection\label{\detokenize{pokemonClient:pokemonClient.displayPokedex}}\pysiglinewithargsret{\sphinxcode{\sphinxupquote{pokemonClient.}}\sphinxbfcode{\sphinxupquote{displayPokedex}}}{\emph{pokedex}}{}~\begin{description}
\item[{Imprime en pantalla el Pokedex de manera}] \leavevmode
«amigable».

\end{description}
\begin{quote}\begin{description}
\item[{Parámetros}] \leavevmode
\sphinxstyleliteralstrong{\sphinxupquote{pokedex}} (\sphinxstyleliteralemphasis{\sphinxupquote{List of String}}) \textendash{} Pokedex de Pokemones

\item[{Devuelve}] \leavevmode
Nada

\end{description}\end{quote}

\end{fulllineitems}

\index{login() (en el módulo pokemonClient)@\spxentry{login()}\spxextra{en el módulo pokemonClient}}

\begin{fulllineitems}
\phantomsection\label{\detokenize{pokemonClient:pokemonClient.login}}\pysiglinewithargsret{\sphinxcode{\sphinxupquote{pokemonClient.}}\sphinxbfcode{\sphinxupquote{login}}}{\emph{soc}}{}
Transfiere los datos al servidor para validar el acceso, y cierra el programa si los datos no son válidos.
\begin{quote}\begin{description}
\item[{Parámetros}] \leavevmode
\sphinxstyleliteralstrong{\sphinxupquote{soc}} (\sphinxstyleliteralemphasis{\sphinxupquote{Socket}}) \textendash{} Socket de la conexión

\item[{Devuelve}] \leavevmode
Nada

\end{description}\end{quote}

\end{fulllineitems}

\index{main() (en el módulo pokemonClient)@\spxentry{main()}\spxextra{en el módulo pokemonClient}}

\begin{fulllineitems}
\phantomsection\label{\detokenize{pokemonClient:pokemonClient.main}}\pysiglinewithargsret{\sphinxcode{\sphinxupquote{pokemonClient.}}\sphinxbfcode{\sphinxupquote{main}}}{}{}
Función principal

\end{fulllineitems}

\index{muestraCatalogo() (en el módulo pokemonClient)@\spxentry{muestraCatalogo()}\spxextra{en el módulo pokemonClient}}

\begin{fulllineitems}
\phantomsection\label{\detokenize{pokemonClient:pokemonClient.muestraCatalogo}}\pysiglinewithargsret{\sphinxcode{\sphinxupquote{pokemonClient.}}\sphinxbfcode{\sphinxupquote{muestraCatalogo}}}{\emph{soc}}{}~\begin{description}
\item[{Le muestra el catálogo disponible de Pokemones}] \leavevmode
al usuario.

\end{description}
\begin{quote}\begin{description}
\item[{Parámetros}] \leavevmode
\sphinxstyleliteralstrong{\sphinxupquote{soc}} (\sphinxstyleliteralemphasis{\sphinxupquote{Socket}}) \textendash{} Socket de la  conexión

\item[{Devuelve}] \leavevmode
Nada

\end{description}\end{quote}

\end{fulllineitems}

\index{muestraPokedex() (en el módulo pokemonClient)@\spxentry{muestraPokedex()}\spxextra{en el módulo pokemonClient}}

\begin{fulllineitems}
\phantomsection\label{\detokenize{pokemonClient:pokemonClient.muestraPokedex}}\pysiglinewithargsret{\sphinxcode{\sphinxupquote{pokemonClient.}}\sphinxbfcode{\sphinxupquote{muestraPokedex}}}{\emph{soc}}{}~\begin{description}
\item[{Muestra el Pokedex del usuario que solicita esta acción}] \leavevmode
al usuario.

\end{description}
\begin{quote}\begin{description}
\item[{Parámetros}] \leavevmode
\sphinxstyleliteralstrong{\sphinxupquote{soc}} (\sphinxstyleliteralemphasis{\sphinxupquote{Socket}}) \textendash{} Socket de la conexión

\item[{Devuelve}] \leavevmode
Nada

\end{description}\end{quote}

\end{fulllineitems}

\index{muestraPokemon() (en el módulo pokemonClient)@\spxentry{muestraPokemon()}\spxextra{en el módulo pokemonClient}}

\begin{fulllineitems}
\phantomsection\label{\detokenize{pokemonClient:pokemonClient.muestraPokemon}}\pysiglinewithargsret{\sphinxcode{\sphinxupquote{pokemonClient.}}\sphinxbfcode{\sphinxupquote{muestraPokemon}}}{\emph{bytes}}{}
Despliega el pokemon asignado.
\begin{quote}\begin{description}
\item[{Parámetros}] \leavevmode
\sphinxstyleliteralstrong{\sphinxupquote{bytes}} (\sphinxstyleliteralemphasis{\sphinxupquote{bytearray}}) \textendash{} bytes de la imagen del pokemon a desplegar

\item[{Devuelve}] \leavevmode
Nada

\end{description}\end{quote}

\end{fulllineitems}

\index{playPokemon() (en el módulo pokemonClient)@\spxentry{playPokemon()}\spxextra{en el módulo pokemonClient}}

\begin{fulllineitems}
\phantomsection\label{\detokenize{pokemonClient:pokemonClient.playPokemon}}\pysiglinewithargsret{\sphinxcode{\sphinxupquote{pokemonClient.}}\sphinxbfcode{\sphinxupquote{playPokemon}}}{\emph{soc}}{}
Permite que el usuario juegue Pokemon Go.
\begin{quote}\begin{description}
\item[{Parámetros}] \leavevmode
\sphinxstyleliteralstrong{\sphinxupquote{soc}} (\sphinxstyleliteralemphasis{\sphinxupquote{Socket}}) \textendash{} Socket de la conexión

\item[{Devuelve}] \leavevmode
Nada

\end{description}\end{quote}

\end{fulllineitems}

\index{printPokemon() (en el módulo pokemonClient)@\spxentry{printPokemon()}\spxextra{en el módulo pokemonClient}}

\begin{fulllineitems}
\phantomsection\label{\detokenize{pokemonClient:pokemonClient.printPokemon}}\pysiglinewithargsret{\sphinxcode{\sphinxupquote{pokemonClient.}}\sphinxbfcode{\sphinxupquote{printPokemon}}}{}{}
Imprime en pantalla el logo Pokemon
:returns: Nada

\end{fulllineitems}

\index{terminarConTimeout() (en el módulo pokemonClient)@\spxentry{terminarConTimeout()}\spxextra{en el módulo pokemonClient}}

\begin{fulllineitems}
\phantomsection\label{\detokenize{pokemonClient:pokemonClient.terminarConTimeout}}\pysiglinewithargsret{\sphinxcode{\sphinxupquote{pokemonClient.}}\sphinxbfcode{\sphinxupquote{terminarConTimeout}}}{\emph{soc}}{}~\begin{description}
\item[{Termina la conexión pues el tiempo de espera de la}] \leavevmode
respuesta del Servidor ha excedido.

\end{description}
\begin{quote}\begin{description}
\item[{Parámetros}] \leavevmode
\sphinxstyleliteralstrong{\sphinxupquote{soc}} (\sphinxstyleliteralemphasis{\sphinxupquote{Socket}}) \textendash{} Socket de la conexión

\item[{Devuelve}] \leavevmode
Nada

\end{description}\end{quote}

\end{fulllineitems}

\index{terminarConexion() (en el módulo pokemonClient)@\spxentry{terminarConexion()}\spxextra{en el módulo pokemonClient}}

\begin{fulllineitems}
\phantomsection\label{\detokenize{pokemonClient:pokemonClient.terminarConexion}}\pysiglinewithargsret{\sphinxcode{\sphinxupquote{pokemonClient.}}\sphinxbfcode{\sphinxupquote{terminarConexion}}}{}{}~\begin{description}
\item[{Termina la conexion pues el Servidor notifica que}] \leavevmode
el tiempo de espera ha excedido.

\end{description}
\begin{quote}\begin{description}
\item[{Param}] \leavevmode
Nada

\item[{Devuelve}] \leavevmode
Nada

\end{description}\end{quote}

\end{fulllineitems}



\section{Servidor Pokemon Go!}
\label{\detokenize{pokemonServer:servidor-pokemon-go}}\label{\detokenize{pokemonServer::doc}}
Implementación de un servidor para el juego Pokemon Go!

\phantomsection\label{\detokenize{pokemonServer:module-pokemonServer}}\index{pokemonServer (módulo)@\spxentry{pokemonServer}\spxextra{módulo}}\index{avisoTimeout() (en el módulo pokemonServer)@\spxentry{avisoTimeout()}\spxextra{en el módulo pokemonServer}}

\begin{fulllineitems}
\phantomsection\label{\detokenize{pokemonServer:pokemonServer.avisoTimeout}}\pysiglinewithargsret{\sphinxcode{\sphinxupquote{pokemonServer.}}\sphinxbfcode{\sphinxupquote{avisoTimeout}}}{\emph{connection}}{}
Manda el mensaje de cierre de sesión al cliente por tiempo de espera excedido.
\begin{quote}\begin{description}
\item[{Parámetros}] \leavevmode
\sphinxstyleliteralstrong{\sphinxupquote{connection}} (\sphinxstyleliteralemphasis{\sphinxupquote{Conexión}}) \textendash{} Conexión entre el cliente y el servidor

\item[{Devuelve}] \leavevmode
Nada

\end{description}\end{quote}

\end{fulllineitems}

\index{cerrarSesion() (en el módulo pokemonServer)@\spxentry{cerrarSesion()}\spxextra{en el módulo pokemonServer}}

\begin{fulllineitems}
\phantomsection\label{\detokenize{pokemonServer:pokemonServer.cerrarSesion}}\pysiglinewithargsret{\sphinxcode{\sphinxupquote{pokemonServer.}}\sphinxbfcode{\sphinxupquote{cerrarSesion}}}{\emph{connection}}{}
Cierre de sesión entre el servidor y el cliente al cual le pertenece la conexión.
\begin{quote}\begin{description}
\item[{Parámetros}] \leavevmode
\sphinxstyleliteralstrong{\sphinxupquote{connection}} (\sphinxstyleliteralemphasis{\sphinxupquote{Conexión}}) \textendash{} Conexión entre el cliente y el servidor

\item[{Devuelve}] \leavevmode
Nada

\end{description}\end{quote}

\end{fulllineitems}

\index{clientThread() (en el módulo pokemonServer)@\spxentry{clientThread()}\spxextra{en el módulo pokemonServer}}

\begin{fulllineitems}
\phantomsection\label{\detokenize{pokemonServer:pokemonServer.clientThread}}\pysiglinewithargsret{\sphinxcode{\sphinxupquote{pokemonServer.}}\sphinxbfcode{\sphinxupquote{clientThread}}}{\emph{connection}, \emph{ip}, \emph{port}, \emph{max\_buffer\_size=5120}}{}
Manejador del hilo que sostiene la conexión entre el servidor y un cliente
\begin{quote}\begin{description}
\item[{Parámetros}] \leavevmode\begin{itemize}
\item {} 
\sphinxstyleliteralstrong{\sphinxupquote{connection}} (\sphinxstyleliteralemphasis{\sphinxupquote{Conexión}}) \textendash{} Conexión entre el servidor y el cliente que abrió el hilo

\item {} 
\sphinxstyleliteralstrong{\sphinxupquote{ip}} (\sphinxstyleliteralemphasis{\sphinxupquote{String}}) \textendash{} Dirección IP de la conexión

\item {} 
\sphinxstyleliteralstrong{\sphinxupquote{port}} (\sphinxstyleliteralemphasis{\sphinxupquote{Integer}}) \textendash{} Puerto a través del cual el servidor mantiene la conexión con el cliente

\item {} 
\sphinxstyleliteralstrong{\sphinxupquote{max\_buffer\_size}} (\sphinxstyleliteralemphasis{\sphinxupquote{Integer}}) \textendash{} Número máximo de bytes que puede recibir en un paquete del cliente

\end{itemize}

\item[{Devuelve}] \leavevmode
Nada

\end{description}\end{quote}

\end{fulllineitems}

\index{getNombrePokemon() (en el módulo pokemonServer)@\spxentry{getNombrePokemon()}\spxextra{en el módulo pokemonServer}}

\begin{fulllineitems}
\phantomsection\label{\detokenize{pokemonServer:pokemonServer.getNombrePokemon}}\pysiglinewithargsret{\sphinxcode{\sphinxupquote{pokemonServer.}}\sphinxbfcode{\sphinxupquote{getNombrePokemon}}}{\emph{idPokemon}}{}
Regresa el nombre del pokemon dado su id.
\begin{quote}\begin{description}
\item[{Parámetros}] \leavevmode
\sphinxstyleliteralstrong{\sphinxupquote{idPokemon}} (\sphinxstyleliteralemphasis{\sphinxupquote{Integer}}) \textendash{} Id del Pokemon a capturado

\item[{Devuelve}] \leavevmode
String

\end{description}\end{quote}

\end{fulllineitems}

\index{giveAccess() (en el módulo pokemonServer)@\spxentry{giveAccess()}\spxextra{en el módulo pokemonServer}}

\begin{fulllineitems}
\phantomsection\label{\detokenize{pokemonServer:pokemonServer.giveAccess}}\pysiglinewithargsret{\sphinxcode{\sphinxupquote{pokemonServer.}}\sphinxbfcode{\sphinxupquote{giveAccess}}}{\emph{connection}, \emph{max\_buffer\_size=5120}}{}
Autentifica a usuarios registrados y proporciona acceso a la ejecución de la aplicación
\begin{quote}\begin{description}
\item[{Parámetros}] \leavevmode\begin{itemize}
\item {} 
\sphinxstyleliteralstrong{\sphinxupquote{connection}} (\sphinxstyleliteralemphasis{\sphinxupquote{Conexión}}) \textendash{} Conexión entre el servidor y el cliente que abrió el hilo

\item {} 
\sphinxstyleliteralstrong{\sphinxupquote{max\_buffer\_size}} (\sphinxstyleliteralemphasis{\sphinxupquote{Integer}}) \textendash{} Número máximo de bytes que puede recibir en un paquete del cliente

\end{itemize}

\item[{Devuelve}] \leavevmode
Integer, String -\textgreater{} Valor que representa la correctud del accesso; Cadena que reprensenta al usuario activo.

\end{description}\end{quote}

\end{fulllineitems}

\index{guardaEnPokedex() (en el módulo pokemonServer)@\spxentry{guardaEnPokedex()}\spxextra{en el módulo pokemonServer}}

\begin{fulllineitems}
\phantomsection\label{\detokenize{pokemonServer:pokemonServer.guardaEnPokedex}}\pysiglinewithargsret{\sphinxcode{\sphinxupquote{pokemonServer.}}\sphinxbfcode{\sphinxupquote{guardaEnPokedex}}}{\emph{idPokemon}, \emph{user}}{}~\begin{description}
\item[{Guarda el pokemon capturado en el pokedex del}] \leavevmode
usuario.

\end{description}
\begin{quote}\begin{description}
\item[{Parámetros}] \leavevmode\begin{itemize}
\item {} 
\sphinxstyleliteralstrong{\sphinxupquote{idPokemon}} (\sphinxstyleliteralemphasis{\sphinxupquote{Integer}}) \textendash{} Id del Pokemon a capturado

\item {} 
\sphinxstyleliteralstrong{\sphinxupquote{user}} (\sphinxstyleliteralemphasis{\sphinxupquote{String}}) \textendash{} Usuario que capturo

\end{itemize}

\item[{Devuelve}] \leavevmode
Nada

\end{description}\end{quote}

\end{fulllineitems}

\index{main() (en el módulo pokemonServer)@\spxentry{main()}\spxextra{en el módulo pokemonServer}}

\begin{fulllineitems}
\phantomsection\label{\detokenize{pokemonServer:pokemonServer.main}}\pysiglinewithargsret{\sphinxcode{\sphinxupquote{pokemonServer.}}\sphinxbfcode{\sphinxupquote{main}}}{}{}
Función principal.

\end{fulllineitems}

\index{muestraCatalogo() (en el módulo pokemonServer)@\spxentry{muestraCatalogo()}\spxextra{en el módulo pokemonServer}}

\begin{fulllineitems}
\phantomsection\label{\detokenize{pokemonServer:pokemonServer.muestraCatalogo}}\pysiglinewithargsret{\sphinxcode{\sphinxupquote{pokemonServer.}}\sphinxbfcode{\sphinxupquote{muestraCatalogo}}}{\emph{connection}}{}
Muestra el catalogo.
\begin{quote}\begin{description}
\item[{Parámetros}] \leavevmode
\sphinxstyleliteralstrong{\sphinxupquote{connection}} (\sphinxstyleliteralemphasis{\sphinxupquote{Conexión}}) \textendash{} Conexión entre el servidor y el cliente

\item[{Devuelve}] \leavevmode
Nada

\end{description}\end{quote}

\end{fulllineitems}

\index{muestraPokedex() (en el módulo pokemonServer)@\spxentry{muestraPokedex()}\spxextra{en el módulo pokemonServer}}

\begin{fulllineitems}
\phantomsection\label{\detokenize{pokemonServer:pokemonServer.muestraPokedex}}\pysiglinewithargsret{\sphinxcode{\sphinxupquote{pokemonServer.}}\sphinxbfcode{\sphinxupquote{muestraPokedex}}}{\emph{connection}, \emph{user}}{}
Muestra el Pokedex del usuario.
\begin{quote}\begin{description}
\item[{Parámetros}] \leavevmode\begin{itemize}
\item {} 
\sphinxstyleliteralstrong{\sphinxupquote{connection}} (\sphinxstyleliteralemphasis{\sphinxupquote{Conexión}}) \textendash{} Conexión entre el servidor y el cliente

\item {} 
\sphinxstyleliteralstrong{\sphinxupquote{user}} (\sphinxstyleliteralemphasis{\sphinxupquote{String}}) \textendash{} Usuario que capturo

\end{itemize}

\item[{Devuelve}] \leavevmode
Nada

\end{description}\end{quote}

\end{fulllineitems}

\index{playPokemonGo() (en el módulo pokemonServer)@\spxentry{playPokemonGo()}\spxextra{en el módulo pokemonServer}}

\begin{fulllineitems}
\phantomsection\label{\detokenize{pokemonServer:pokemonServer.playPokemonGo}}\pysiglinewithargsret{\sphinxcode{\sphinxupquote{pokemonServer.}}\sphinxbfcode{\sphinxupquote{playPokemonGo}}}{\emph{connection}, \emph{user}}{}
Método que simula el comportamiento del juego Pokemon Go.
\begin{quote}\begin{description}
\item[{Parámetros}] \leavevmode
\sphinxstyleliteralstrong{\sphinxupquote{connection}} (\sphinxstyleliteralemphasis{\sphinxupquote{Conexión}}) \textendash{} Conexión entre el servidor y el cliente

\item[{Devuelve}] \leavevmode
Nada

\end{description}\end{quote}

\end{fulllineitems}

\index{start\_server() (en el módulo pokemonServer)@\spxentry{start\_server()}\spxextra{en el módulo pokemonServer}}

\begin{fulllineitems}
\phantomsection\label{\detokenize{pokemonServer:pokemonServer.start_server}}\pysiglinewithargsret{\sphinxcode{\sphinxupquote{pokemonServer.}}\sphinxbfcode{\sphinxupquote{start\_server}}}{}{}
Inicialización del servidor
\begin{quote}\begin{description}
\item[{Parámetros}] \leavevmode
\sphinxstyleliteralstrong{\sphinxupquote{ip\_dir}} (\sphinxstyleliteralemphasis{\sphinxupquote{String}}) \textendash{} Dirección IP del socket al cual se va conectar el servidor

\item[{Devuelve}] \leavevmode
Nada

\end{description}\end{quote}

\end{fulllineitems}

\index{terminarConexion() (en el módulo pokemonServer)@\spxentry{terminarConexion()}\spxextra{en el módulo pokemonServer}}

\begin{fulllineitems}
\phantomsection\label{\detokenize{pokemonServer:pokemonServer.terminarConexion}}\pysiglinewithargsret{\sphinxcode{\sphinxupquote{pokemonServer.}}\sphinxbfcode{\sphinxupquote{terminarConexion}}}{}{}~\begin{description}
\item[{Termina la conexion pues el Cliente notifica que el tiempo}] \leavevmode
de espera ha excedido.

\end{description}
\begin{quote}\begin{description}
\item[{Param}] \leavevmode
Nada

\item[{Devuelve}] \leavevmode
Nada

\end{description}\end{quote}

\end{fulllineitems}



\renewcommand{\indexname}{Índice de Módulos Python}
\begin{sphinxtheindex}
\let\bigletter\sphinxstyleindexlettergroup
\bigletter{p}
\item\relax\sphinxstyleindexentry{pokemonClient}\sphinxstyleindexpageref{pokemonClient:\detokenize{module-pokemonClient}}
\item\relax\sphinxstyleindexentry{pokemonServer}\sphinxstyleindexpageref{pokemonServer:\detokenize{module-pokemonServer}}
\end{sphinxtheindex}

\renewcommand{\indexname}{Índice}
\printindex
\end{document}