%% Generated by Sphinx.
\def\sphinxdocclass{report}
\documentclass[letterpaper,10pt,spanish,openany,oneside]{sphinxmanual}
\ifdefined\pdfpxdimen
   \let\sphinxpxdimen\pdfpxdimen\else\newdimen\sphinxpxdimen
\fi \sphinxpxdimen=.75bp\relax

\PassOptionsToPackage{warn}{textcomp}
\usepackage[utf8]{inputenc}
\ifdefined\DeclareUnicodeCharacter
% support both utf8 and utf8x syntaxes
  \ifdefined\DeclareUnicodeCharacterAsOptional
    \def\sphinxDUC#1{\DeclareUnicodeCharacter{"#1}}
  \else
    \let\sphinxDUC\DeclareUnicodeCharacter
  \fi
  \sphinxDUC{00A0}{\nobreakspace}
  \sphinxDUC{2500}{\sphinxunichar{2500}}
  \sphinxDUC{2502}{\sphinxunichar{2502}}
  \sphinxDUC{2514}{\sphinxunichar{2514}}
  \sphinxDUC{251C}{\sphinxunichar{251C}}
  \sphinxDUC{2572}{\textbackslash}
\fi
\usepackage{cmap}
\usepackage[T1]{fontenc}
\usepackage{amsmath,amssymb,amstext}
\usepackage{babel}



\usepackage{times}
\expandafter\ifx\csname T@LGR\endcsname\relax
\else
% LGR was declared as font encoding
  \substitutefont{LGR}{\rmdefault}{cmr}
  \substitutefont{LGR}{\sfdefault}{cmss}
  \substitutefont{LGR}{\ttdefault}{cmtt}
\fi
\expandafter\ifx\csname T@X2\endcsname\relax
  \expandafter\ifx\csname T@T2A\endcsname\relax
  \else
  % T2A was declared as font encoding
    \substitutefont{T2A}{\rmdefault}{cmr}
    \substitutefont{T2A}{\sfdefault}{cmss}
    \substitutefont{T2A}{\ttdefault}{cmtt}
  \fi
\else
% X2 was declared as font encoding
  \substitutefont{X2}{\rmdefault}{cmr}
  \substitutefont{X2}{\sfdefault}{cmss}
  \substitutefont{X2}{\ttdefault}{cmtt}
\fi


\usepackage[Sonny]{fncychap}
\ChNameVar{\Large\normalfont\sffamily}
\ChTitleVar{\Large\normalfont\sffamily}
\usepackage{sphinx}

\fvset{fontsize=\small}
\usepackage{geometry}

% Include hyperref last.
\usepackage{hyperref}
% Fix anchor placement for figures with captions.
\usepackage{hypcap}% it must be loaded after hyperref.
% Set up styles of URL: it should be placed after hyperref.
\urlstyle{same}
\addto\captionsspanish{\renewcommand{\contentsname}{Contents:}}

\usepackage{sphinxmessages}
\setcounter{tocdepth}{1}



\title{Proyecto2-Pokemon}
\date{08 de diciembre de 2019}
\release{}
\author{doggos}
\newcommand{\sphinxlogo}{\vbox{}}
\renewcommand{\releasename}{}
\makeindex
\begin{document}

\ifdefined\shorthandoff
  \ifnum\catcode`\=\string=\active\shorthandoff{=}\fi
  \ifnum\catcode`\"=\active\shorthandoff{"}\fi
\fi

\pagestyle{empty}
\sphinxmaketitle
\pagestyle{plain}
\sphinxtableofcontents
\pagestyle{normal}
\phantomsection\label{\detokenize{index::doc}}


\begin{sphinxadmonition}{note}{Nota:}
Bienvenido a la documentación del Proyecto 2 de la materia de Redes de Computadoras
\end{sphinxadmonition}


\chapter{Programas involucrados}
\label{\detokenize{index:programas-involucrados}}

\section{Cliente}
\label{\detokenize{client:cliente}}\label{\detokenize{client::doc}}
Modela a un cliente del juego Pokemon Go! e interactúa directamente con el usuario

\phantomsection\label{\detokenize{client:module-client}}\index{client (módulo)@\spxentry{client}\spxextra{módulo}}\index{cerrarSesion() (en el módulo client)@\spxentry{cerrarSesion()}\spxextra{en el módulo client}}

\begin{fulllineitems}
\phantomsection\label{\detokenize{client:client.cerrarSesion}}\pysiglinewithargsret{\sphinxcode{\sphinxupquote{client.}}\sphinxbfcode{\sphinxupquote{cerrarSesion}}}{\emph{soc}}{}
Cierre de sesión del usuario
\begin{quote}\begin{description}
\item[{Parámetros}] \leavevmode
\sphinxstyleliteralstrong{\sphinxupquote{soc}} (\sphinxstyleliteralemphasis{\sphinxupquote{Socket}}) \textendash{} Socket de la conexión

\item[{Devuelve}] \leavevmode
Nada

\end{description}\end{quote}

\end{fulllineitems}

\index{login() (en el módulo client)@\spxentry{login()}\spxextra{en el módulo client}}

\begin{fulllineitems}
\phantomsection\label{\detokenize{client:client.login}}\pysiglinewithargsret{\sphinxcode{\sphinxupquote{client.}}\sphinxbfcode{\sphinxupquote{login}}}{\emph{soc}}{}
Transfiere los datos al servidor para validar el acceso, y cierra el programa si los datos no son válidos
\begin{quote}\begin{description}
\item[{Parámetros}] \leavevmode
\sphinxstyleliteralstrong{\sphinxupquote{soc}} (\sphinxstyleliteralemphasis{\sphinxupquote{Socket}}) \textendash{} Socket de la conexión

\item[{Devuelve}] \leavevmode
Nada

\end{description}\end{quote}

\end{fulllineitems}

\index{main() (en el módulo client)@\spxentry{main()}\spxextra{en el módulo client}}

\begin{fulllineitems}
\phantomsection\label{\detokenize{client:client.main}}\pysiglinewithargsret{\sphinxcode{\sphinxupquote{client.}}\sphinxbfcode{\sphinxupquote{main}}}{}{}
Función principal

\end{fulllineitems}

\index{playPokemon() (en el módulo client)@\spxentry{playPokemon()}\spxextra{en el módulo client}}

\begin{fulllineitems}
\phantomsection\label{\detokenize{client:client.playPokemon}}\pysiglinewithargsret{\sphinxcode{\sphinxupquote{client.}}\sphinxbfcode{\sphinxupquote{playPokemon}}}{\emph{soc}}{}
Permite que el usuario juegue Pokemon Go
\begin{quote}\begin{description}
\item[{Parámetros}] \leavevmode
\sphinxstyleliteralstrong{\sphinxupquote{soc}} (\sphinxstyleliteralemphasis{\sphinxupquote{Socket}}) \textendash{} Socket de la conexión

\item[{Devuelve}] \leavevmode
Nada

\end{description}\end{quote}

\end{fulllineitems}



\section{Servidor}
\label{\detokenize{server:servidor}}\label{\detokenize{server::doc}}
Modela al servidor del juego Pokemon Go!

\phantomsection\label{\detokenize{server:module-server}}\index{server (módulo)@\spxentry{server}\spxextra{módulo}}\index{cerrarSesion() (en el módulo server)@\spxentry{cerrarSesion()}\spxextra{en el módulo server}}

\begin{fulllineitems}
\phantomsection\label{\detokenize{server:server.cerrarSesion}}\pysiglinewithargsret{\sphinxcode{\sphinxupquote{server.}}\sphinxbfcode{\sphinxupquote{cerrarSesion}}}{\emph{connection}}{}
Cierre de sesión entre el servidor y el cliente al cual le pertenece la conexión
\begin{quote}\begin{description}
\item[{Parámetros}] \leavevmode
\sphinxstyleliteralstrong{\sphinxupquote{- conexión entre el cliente y el servidor}} (\sphinxstyleliteralemphasis{\sphinxupquote{connection}}) \textendash{} 

\end{description}\end{quote}

\end{fulllineitems}

\index{clientThread() (en el módulo server)@\spxentry{clientThread()}\spxextra{en el módulo server}}

\begin{fulllineitems}
\phantomsection\label{\detokenize{server:server.clientThread}}\pysiglinewithargsret{\sphinxcode{\sphinxupquote{server.}}\sphinxbfcode{\sphinxupquote{clientThread}}}{\emph{connection}, \emph{ip}, \emph{port}, \emph{max\_buffer\_size=5120}}{}
Manejador del hilo que sostiene la conexión entre el servidor y un cliente
\begin{quote}\begin{description}
\item[{Parámetros}] \leavevmode\begin{itemize}
\item {} 
\sphinxstyleliteralstrong{\sphinxupquote{connection}} (\sphinxstyleliteralemphasis{\sphinxupquote{Conexión}}) \textendash{} Conexión entre el servidor y el cliente que abrió el hilo

\item {} 
\sphinxstyleliteralstrong{\sphinxupquote{ip}} (\sphinxstyleliteralemphasis{\sphinxupquote{Cadena}}) \textendash{} Dirección IP de la conexión

\item {} 
\sphinxstyleliteralstrong{\sphinxupquote{port}} (\sphinxstyleliteralemphasis{\sphinxupquote{Entero}}) \textendash{} Puerto a través del cual el servidor mantiene la conexión con el cliente

\item {} 
\sphinxstyleliteralstrong{\sphinxupquote{max\_buffer\_size}} (\sphinxstyleliteralemphasis{\sphinxupquote{Entero}}) \textendash{} Número máximo de bytes que puede recibir en un paquete del cliente

\end{itemize}

\end{description}\end{quote}

\end{fulllineitems}

\index{giveAccess() (en el módulo server)@\spxentry{giveAccess()}\spxextra{en el módulo server}}

\begin{fulllineitems}
\phantomsection\label{\detokenize{server:server.giveAccess}}\pysiglinewithargsret{\sphinxcode{\sphinxupquote{server.}}\sphinxbfcode{\sphinxupquote{giveAccess}}}{\emph{connection}, \emph{max\_buffer\_size=5120}}{}
Autentifica a usuarios registrados y proporciona acceso a la ejecución de la aplicación
\begin{quote}\begin{description}
\item[{Parámetros}] \leavevmode\begin{itemize}
\item {} 
\sphinxstyleliteralstrong{\sphinxupquote{connection}} (\sphinxstyleliteralemphasis{\sphinxupquote{Conexión}}) \textendash{} Conexión entre el servidor y el cliente que abrió el hilo

\item {} 
\sphinxstyleliteralstrong{\sphinxupquote{max\_buffer\_size}} (\sphinxstyleliteralemphasis{\sphinxupquote{Entero}}) \textendash{} Número máximo de bytes que puede recibir en un paquete del cliente

\end{itemize}

\item[{Devuelve}] \leavevmode
int - Indicador de acceso permitido

\end{description}\end{quote}

\end{fulllineitems}

\index{main() (en el módulo server)@\spxentry{main()}\spxextra{en el módulo server}}

\begin{fulllineitems}
\phantomsection\label{\detokenize{server:server.main}}\pysiglinewithargsret{\sphinxcode{\sphinxupquote{server.}}\sphinxbfcode{\sphinxupquote{main}}}{}{}
Función principal.

\end{fulllineitems}

\index{playPokemonGo() (en el módulo server)@\spxentry{playPokemonGo()}\spxextra{en el módulo server}}

\begin{fulllineitems}
\phantomsection\label{\detokenize{server:server.playPokemonGo}}\pysiglinewithargsret{\sphinxcode{\sphinxupquote{server.}}\sphinxbfcode{\sphinxupquote{playPokemonGo}}}{\emph{connection}}{}
Método que simula el comportamiento del juego Pokemon Go
\begin{quote}\begin{description}
\item[{Parámetros}] \leavevmode
\sphinxstyleliteralstrong{\sphinxupquote{- Conexión entre el servidor y el cliente}} (\sphinxstyleliteralemphasis{\sphinxupquote{connection}}) \textendash{} 

\item[{Devuelve}] \leavevmode
None

\end{description}\end{quote}

\end{fulllineitems}

\index{start\_server() (en el módulo server)@\spxentry{start\_server()}\spxextra{en el módulo server}}

\begin{fulllineitems}
\phantomsection\label{\detokenize{server:server.start_server}}\pysiglinewithargsret{\sphinxcode{\sphinxupquote{server.}}\sphinxbfcode{\sphinxupquote{start\_server}}}{}{}
Inicialización del servidor
\begin{quote}\begin{description}
\item[{Devuelve}] \leavevmode
Nada

\end{description}\end{quote}

\end{fulllineitems}



\chapter{¿Cómo usar?}
\label{\detokenize{index:como-usar}}\begin{itemize}
\item {} 
Para el servidor
\begin{itemize}
\item {} 
Pasos previos para instalar la base de datos. Revisar archivo.

\item {} 
En una terminal, nos situamos en la ubicación del archivo \sphinxstyleemphasis{server.py}

\item {} 
El programa no requiere parámetros adicionales para su funcionamiento. Ejecutamos en terminal \sphinxstyleemphasis{./server.py}

\end{itemize}

\item {} 
Para el cliente
\begin{itemize}
\item {} 
En una terminal, nos situamos en la ubicación del archivo \sphinxstyleemphasis{client.py}

\item {} 
Este programa recibe como parámetros iniciales la dirección IP a través de la cual se quiere conectar, y el puerto. Por lo tanto, ejecutamos de la siguiente manera: \sphinxstyleemphasis{./client.py \textless{}IP\textgreater{} \textless{}port\textgreater{}}

\end{itemize}

\end{itemize}


\renewcommand{\indexname}{Índice de Módulos Python}
\begin{sphinxtheindex}
\let\bigletter\sphinxstyleindexlettergroup
\bigletter{c}
\item\relax\sphinxstyleindexentry{client}\sphinxstyleindexpageref{client:\detokenize{module-client}}
\indexspace
\bigletter{s}
\item\relax\sphinxstyleindexentry{server}\sphinxstyleindexpageref{server:\detokenize{module-server}}
\end{sphinxtheindex}

\renewcommand{\indexname}{Índice}
\printindex
\end{document}